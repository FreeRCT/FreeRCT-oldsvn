\documentclass{article}
% $Id$
% This file is part of FreeRCT.
% FreeRCT is free software; you can redistribute it and/or modify it under the terms of the GNU General Public License as published by the Free Software Foundation, version 2.
% FreeRCT is distributed in the hope that it will be useful, but WITHOUT ANY WARRANTY; without even the implied warranty of MERCHANTABILITY or FITNESS FOR A PARTICULAR PURPOSE.
% See the GNU General Public License for more details. You should have received a copy of the GNU General Public License along with FreeRCT. If not, see <http://www.gnu.org/licenses/>.
%
\usepackage[a4paper]{geometry}
\title{File format of the \texttt{$*$.rcd} data files}
\author{Alberth}
\date{Built \today}

\begin{document}
\maketitle
\tableofcontents

%%
%%
%%
\section{File header}
Each data file starts with a file header indicating it is a RCD{} file.
The format is as follows

\begin{center}
\begin{tabular}{|r|c|l|} \hline
\textbf{Offset} & \textbf{Length} & \textbf{Contents description} \\ \hline
0 & 4 & Magic string `RCDF' \\
4 & 4 & Value `1', version number of the data file format. \\ \hline
8 & \multicolumn{2}{l|}{Total length} \\ \hline
\end{tabular}
\end{center}

%%
%%
%%
\section{Data blocks}
After the file header come the various data blocks.
The goal of data blocks is to provide blobs of information that are somewhat independent.
The data blocks are referenced by game blocks by their ID. The first data block
gets number 1, the second block number 2, etc.

A reference to data block 0 means `not present'.


\subsection{Sprite Pixels}
A data block containing the actual image of a sprite (in 8bpp).

\begin{center}
\begin{tabular}{|r|c|l|} \hline
\textbf{Offset} & \textbf{Length} & \textbf{Contents description} \\ \hline
 0 & 4 & Magic string `8PXL' \\
 4 & 4 & Version number of the block `1'. \\
 8 & 4 & Length of the block excluding magic string, version, and length. \\
12 & 2 & Width of the image. \\
14 & 2 & $h$, height of the image. \\
16 & $4*h$ & Jump table to pixel data of each line. Offset is relative to the first \\
   &       & entry of the jump table. Value 0 means there is no data for that line. \\
 ? &    ?  & Pixels of each line. \\ \hline
 ? & \multicolumn{2}{l|}{Variable length} \\ \hline
\end{tabular}
\end{center}

Line data is a sequence of pixels with an offset. Its format is

\begin{center}
\begin{tabular}{|r|c|l|} \hline
\textbf{Offset} & \textbf{Length} & \textbf{Contents description} \\ \hline
 0 & 1 & Relative offset (0-127), bit 7 means `last entry of the line'. \\
 1 & 1 & $n$, number of pixels that follow this count (0-255). \\
 2 & $n$ & Pixels, 1 byte per pixel (as it is 8bpp). \\ \hline
 ? & \multicolumn{2}{l|}{Variable length} \\ \hline
\end{tabular}
\end{center}

The offset byte is relative to the end of the previous pixels, thus an offset
of 0 means no gap between the pixels. A count of 0 is useful if the gap at a
line is longer than 127 pixels.

\medskip
Note: Some simple form of compressing may be useful in the pixels as it
decreases the amount of memory transfers.


\subsection{Sprite block}
Data of a single sprite.

\begin{center}
\begin{tabular}{|r|c|l|} \hline
\textbf{Offset} & \textbf{Length} & \textbf{Contents description} \\ \hline
   0 &  4 & Magic string `SPRT' \\
   4 &  4 & Version number of the block `2'. \\
   8 &  4 & Length of the block excluding magic string, version, and length. \\
  12 &  2 & (signed) X-offset. \\
  14 &  2 & (signed) Y-offset. \\
  16 &  4 & Sprite image data. \\
  20 &  0 & (removed in version 2) Palette data. \\ \hline
  20 & \multicolumn{2}{l|}{Total length} \\ \hline
\end{tabular}
\end{center}


\section{Game blocks}
A game block is a piece of data useful for the game. Normally it refers to one
or more data blocks.

\subsection{Tile surface sprite sub-block}
In several game blocks you can find a set of sprite for the ground. Below is
the layout of such a sub-block.
Note that the sprites should look to the north (thus, the sprite at 4 has its
back corner up).

\begin{center}
\begin{tabular}{|r|c|l|} \hline
\textbf{Offset} & \textbf{Length} & \textbf{Contents description} \\ \hline
   0 &  4 & Flat surface tile. \\
   4 &  4 & North corner up. \\
   8 &  4 & East corner up. \\
  12 &  4 & North, east corners up. \\
  16 &  4 & South corner up. \\
  20 &  4 & North, south corners up. \\
  24 &  4 & East, south corners up. \\
  28 &  4 & North, east, south corners up. \\
  32 &  4 & West, north corners up. \\
  36 &  4 & West, east corners up. \\
  40 &  4 & West, north, east corners up. \\
  44 &  4 & West, south corners up. \\
  48 &  4 & West, north, south corners up. \\
  52 &  4 & West, east, south corners up. \\
  56 &  4 & Steep north slope. \\
  60 &  4 & Steep east slope. \\
  64 &  4 & Steep south slope. \\
  68 &  4 & Steep west slope. \\ \hline
  72 & \multicolumn{2}{l|}{Total length of the sub-block} \\ \hline
\end{tabular}
\end{center}


\subsection{Ground tiles block}
A set of ground tiles that form a smooth surface.

\begin{center}
\begin{tabular}{|r|c|l|} \hline
\textbf{Offset} & \textbf{Length} & \textbf{Contents description} \\ \hline
   0 &  4 & Magic string `SURF' \\
   4 &  4 & Version number of the block `3'. \\
   8 &  4 & Length of the block excluding magic string, version, and length. \\
  12 &  2 & (added in version 2) Type of ground. \\
  14 &  2 & Width of a tile of the surface. \\
  16 &  2 & Change in Z height (in pixels) when going up or down a tile level. \\
  18 & 72 & Tile surface sprite sub-block for north viewing direction. \\
  90 &  0 & (removed in version 3) Tile surface sprite sub-block for east viewing direction. \\
  90 &  0 & (removed in version 3) Tile surface sprite sub-block for south viewing direction. \\
  90 &  0 & (removed in version 3) Tile surface sprite sub-block for west viewing direction. \\ \hline
  90 & \multicolumn{2}{l|}{Total length} \\ \hline
\end{tabular}
\end{center}

\medskip
\noindent
Known types of ground:
\begin{itemize}
\item Empty (0) Reserved, do not use in the RCD{} file.
\item Grass (16-19) Green grass ground, with increasing length grass on it.
\item Sand  (32) Desert `ground'.
\item Cursor (48) Cursor test tiles. Internal use. Defines what part of a tile
    is selected. Colour 181 means 'north corner', 182 means 'east corner', 184
    means 'west corner', 185 means 'south corner', and 183 means 'entire
    tile'.
\end{itemize}

\textbf{To do: Move the cursor tile to another position.}

%%
%%
%%
\subsection{Tile selection}
A tile selection cursor. It is very similar to ground tiles, except there is
no type.

\begin{center}
\begin{tabular}{|r|c|l|} \hline
\textbf{Offset} & \textbf{Length} & \textbf{Contents description} \\ \hline
   0 &  4 & Magic string `TSEL' \\
   4 &  4 & Version number of the block `1'. \\
   8 &  4 & Length of the block excluding magic string, version, and length. \\
  12 &  2 & Width of a tile of the surface. \\
  14 &  2 & Change in Z height (in pixels) when going up or down a tile level. \\
  16 & 72 & Tile surface sprite sub-block. \\ \hline
  88 & \multicolumn{2}{l|}{Total length} \\ \hline
\end{tabular}
\end{center}

\subsection{Tile area selection}
\begin{center}
\begin{tabular}{|r|c|l|} \hline
\textbf{Offset} & \textbf{Length} & \textbf{Contents description} \\ \hline
   0 &  4 & Magic string `TARE' \\
   4 &  4 & Version number of the block `1'. \\
   8 &  4 & Length of the block excluding magic string, version, and length. \\
  12 &  2 & Width of a tile of the surface. \\
  14 &  2 & Change in Z height (in pixels) when going up or down a tile level. \\
  16 & 72 & Tile surface sprite sub-block. \\ \hline
  88 & \multicolumn{2}{l|}{Total length} \\ \hline
\end{tabular}
\end{center}


\subsection{Patrol area selection}
\begin{center}
\begin{tabular}{|r|c|l|} \hline
\textbf{Offset} & \textbf{Length} & \textbf{Contents description} \\ \hline
   0 &  4 & Magic string `PARE' \\
   4 &  4 & Version number of the block `1'. \\
   8 &  4 & Length of the block excluding magic string, version, and length. \\
  12 &  2 & Width of a tile of the surface. \\
  14 &  2 & Change in Z height (in pixels) when going up or down a tile level. \\
  16 & 72 & Tile surface sprite sub-block. \\ \hline
  88 & \multicolumn{2}{l|}{Total length} \\ \hline
\end{tabular}
\end{center}


\subsection{Tile corner selection block}
\begin{center}
\begin{tabular}{|r|c|l|} \hline
\textbf{Offset} & \textbf{Length} & \textbf{Contents description} \\ \hline
   0 &  4 & Magic string `TCOR' \\
   4 &  4 & Version number of the block `1'. \\
   8 &  4 & Length of the block excluding magic string, version, and length. \\
  12 &  2 & Width of a tile of the surface. \\
  14 &  2 & Change in Z height (in pixels) when going up or down a tile level. \\
  16 & 72 & Tile surface sprite sub-block for selected corner pointing north. \\
  88 & 72 & Tile surface sprite sub-block for selected corner pointing east. \\
 160 & 72 & Tile surface sprite sub-block for selected corner pointing south. \\
 232 & 72 & Tile surface sprite sub-block for selected corner pointing west. \\ \hline
 304 & \multicolumn{2}{l|}{Total length} \\ \hline
\end{tabular}
\end{center}


\subsection{Shops/stalls}
One tile objects.

\begin{center}
\begin{tabular}{|r|c|l|} \hline
\textbf{Offset} & \textbf{Length} & \textbf{Contents description} \\ \hline
   0 &  4 & Magic string `SHOP' \\
   4 &  4 & Version number of the block `1'. \\
   8 &  4 & Length of the block excluding magic string, version, and length. \\
  12 &  2 & Width of a tile of the surface. \\
  14 &  2 & Height of the shop in voxels. \\
  16 &  4 & View to the north where the entrance is at the NE edge. \\
  20 &  4 & View to the north where the entrance is at the SE edge. \\
  24 &  4 & View to the north where the entrance is at the SW edge. \\
  28 &  4 & View to the north where the entrance is at the NW edge. \\ \hline
  32 & \multicolumn{2}{l|}{Total length} \\ \hline
\end{tabular}
\end{center}


\subsection{Build direction arrows}
\begin{center}
\begin{tabular}{|r|c|l|} \hline
\textbf{Offset} & \textbf{Length} & \textbf{Contents description} \\ \hline
   0 &  4 & Magic string `BDIR' \\
   4 &  4 & Version number of the block `1'. \\
   8 &  4 & Length of the block excluding magic string, version, and length. \\
  12 &  2 & Width of a tile of the surface. \\
  14 &  4 & Arrow pointing to NE edge. \\
  18 &  4 & Arrow pointing to SE edge. \\
  22 &  4 & Arrow pointing to SW edge. \\
  26 &  4 & Arrow pointing to NW edge. \\ \hline
  30 & \multicolumn{2}{l|}{Total length} \\ \hline
\end{tabular}
\end{center}


%%
%%
%%
\subsection{Foundations block}
Vertical foundations to close gaps in the smooth surface.

\begin{center}
\begin{tabular}{|r|c|l|} \hline
\textbf{Offset} & \textbf{Length} & \textbf{Contents description} \\ \hline
  0 &  4 & Magic string `FUND' \\
  4 &  4 & Version number of the block `1'. \\
  8 &  4 & Length of the block excluding magic string, version, and length. \\
 12 &  2 & Type of foundation. \\
 14 &  2 & Width of a tile. \\
 16 &  2 & Change in Z height of the tiles. \\
 18 &  4 & Vertical south-east foundation, east up, south down. \\
 22 &  4 & Vertical south-east foundation, east down, south up. \\
 26 &  4 & Vertical south-east foundation, east up, south up. \\
 30 &  4 & Vertical south-west foundation, south up, west down. \\
 34 &  4 & Vertical south-west foundation, south down, west up. \\
 38 &  4 & Vertical south-west foundation, south up, west up. \\ \hline
 42 & \multicolumn{2}{l|}{Total length} \\ \hline
\end{tabular}
\end{center}

\medskip
\noindent
Known types of foundation:
\begin{itemize}
\item Empty (0) Reserved, do not use in the RCD{} file.
\item Ground (16)
\item Wood (32)
\item Brick (48)
\end{itemize}

The tile width and z-height are used to ensure the foundations match with the
surface tiles.

\subsection{Path block}
Path coverage is a set of at most 47 flat images. Paths can connect to
neighbouring tiles through four edges, optionally also covering the corner
between two connecting edges.

Starting at offset 14 are the sprite block numbers of each sprite. As normal,
use 0 to denote absence of a sprite. Two letter words in the description
denote an edge connects, one letter words denote the corner is covered.

Besides the maximal 47 flat sprites there are also 4 sprites with one edge raised.

\medskip
\noindent
Known types of path surface:
\begin{itemize}
\item Empty (0) Reserved, do not use in the RCD{} file.
\item Concrete (16)
\end{itemize}

\begin{center}
\begin{tabular}{|r|c|l|} \hline
\textbf{Offset} & \textbf{Length} & \textbf{Contents description} \\ \hline
   0 &  4 & Magic string `PATH' \\
   4 &  4 & Version number of the block `1'. \\
   8 &  4 & Length of the block excluding magic string, version, and length. \\
  12 &  2 & Type of path surface. \\
  14 &  2 & Width of a tile. \\
  16 &  2 & Change in Z height of the tiles. \\
  18 &  4 & - (empty) \\
  22 &  4 & NE \\
  26 &  4 & SE \\
  30 &  4 & NE, SE \\
  34 &  4 & NE, SE, E \\
  38 &  4 & SW \\
  42 &  4 & NE, SW \\
  46 &  4 & SE, SW \\
  50 &  4 & SE, SW, S \\
  54 &  4 & NE, SE, SW \\
  58 &  4 & NE, SE, SW, E \\
  62 &  4 & NE, SE, SW, S \\
  66 &  4 & NE, SE, SW, E, S \\
  70 &  4 & NW \\
  74 &  4 & NE, NW \\
  78 &  4 & NE, NW, N \\
  82 &  4 & NW, SE \\
  86 &  4 & NE, NW, SE \\
  90 &  4 & NE, NW, SE, N \\
  94 &  4 & NE, NW, SE, E \\
  98 &  4 & NE, NW, SE, N, E \\
 102 &  4 & NW, SW \\
 106 &  4 & NW, SW, W \\
 110 &  4 & NE, NW, SW \\
 114 &  4 & NE, NW, SW, N \\
 118 &  4 & NE, NW, SW, W \\
 122 &  4 & NE, NW, SW, N, W \\
 126 &  4 & NW, SE, SW \\
 130 &  4 & NW, SE, SW, S \\
 134 &  4 & NW, SE, SW, W \\
 138 &  4 & NW, SE, SW, S, W \\
 142 &  4 & NE, NW, SE, SW \\
 146 &  4 & NE, NW, SE, SW, N \\
 150 &  4 & NE, NW, SE, SW, E \\
 154 &  4 & NE, NW, SE, SW, N, E \\
 158 &  4 & NE, NW, SE, SW, S \\
 162 &  4 & NE, NW, SE, SW, N, S \\
 166 &  4 & NE, NW, SE, SW, E, S \\
 170 &  4 & NE, NW, SE, SW, N, E, S \\
 174 &  4 & NE, NW, SE, SW, W \\
\multicolumn{3}{|l|}{\textit{Table continued at next page}} \\ \hline
\end{tabular}
\end{center}

\begin{center}
\begin{tabular}{|r|c|l|} \hline
\multicolumn{3}{|l|}{\textit{Path sprites continued}} \\
\textbf{Offset} & \textbf{Length} & \textbf{Contents description} \\ \hline
 178 &  4 & NE, NW, SE, SW, N, W \\
 182 &  4 & NE, NW, SE, SW, E, W \\
 186 &  4 & NE, NW, SE, SW, N, E, W \\
 190 &  4 & NE, NW, SE, SW, S, W \\
 194 &  4 & NE, NW, SE, SW, N, S, W \\
 198 &  4 & NE, NW, SE, SW, E, S, W \\
 202 &  4 & NE, NW, SE, SW, N, E, S, W \\
 206 &  4 & NE edge up. \\
 210 &  4 & NW edge up. \\
 214 &  4 & SE edge up. \\
 218 &  4 & SW edge up. \\ \hline
 222 & \multicolumn{2}{l|}{Length of one view direction.} \\ \hline
\end{tabular}
\end{center}

\subsection{Platforms}
\begin{center}
\begin{tabular}{|r|c|l|} \hline
\textbf{Offset} & \textbf{Length} & \textbf{Contents description} \\ \hline
   0 &  4 & Magic string `PLAT' \\
   4 &  4 & Version number of the block `1'. \\
   8 &  4 & Length of the block excluding magic string, version, and length. \\
  12 &  2 & Width of a tile of the surface. \\
  14 &  2 & Change in Z height (in pixels) when going up or down a tile level. \\
  16 &  2 & Platform type. \\
  18 &  4 & Flat platform for north and south view. \\
  22 &  4 & Flat platform for east and west view. \\
  26 &  4 & Platform is raised at the NE edge. \\
  30 &  4 & Platform is raised at the SE edge. \\
  34 &  4 & Platform is raised at the SW edge. \\
  38 &  4 & Platform is raised at the NW edge. \\ \hline
  42 & \multicolumn{2}{l|}{Total length} \\ \hline
\end{tabular}
\end{center}

Platform type:
\begin{itemize}
\item Empty 0, do not use.
\item Wood 16.
\end{itemize}


\subsection{Platform supports}
-rotation?
-steep slopes?

%%
%%
%%
\section{GUI}
GUI{} sprites, in various forms.

\subsection{Generic GUI{} border sprites}
The most common form of a widget is a rectangular shape.
To draw such a shape, nine sprites are needed around the border of the
rectangle.

\smallskip
\begin{center}
\begin{tabular}{ccc}
top-left    & top-middle    & top-right \\
left        & middle        & right \\
bottom-left & bottom-middle & bottom-right
\end{tabular}
\end{center}

The `top-left', `top-right', `bottom-left' and `bottom-right' sprites are used
for the corners of the widget or window. The `top-middle', `middle', and
`bottom-middle' should be equally wide, and are used to insert horizontal
space between the left and the right part (with step size equal to the width
of the sprites. The `left', `middle', and `right' do the same, except their
common height is used for vertical resizing.

Except for the `top-left' sprite any of the sprites can be dropped. If you leave out
`top-middle', `middle', or `bottom-middle', horizontal resizing is not
possible. If you leave out `left', `middle', or `right' vertical resizing is
not possible.
If you leave out `top-right', the `top-right', `right', and `bottom-right'
sprites are considered not needed. Similarly for the `bottom-left' sprite.
Supplying the `top-right' sprite but leaving out `bottom-right' (and similarly
for `bottom-left' and `bottom-right') gives undefined behaviour.

A sprite coverage of the edge has four border width parameters (top, left,
right, and bottom), measured in pixels.
In addition, a horizontal and a vertical
offset needs to be specified relative to the bounding box of the widget
contents.

That leads to the following block:
\begin{center}
\begin{tabular}{|r|c|l|} \hline
\textbf{Offset} & \textbf{Length} & \textbf{Contents description} \\ \hline
   0 &  4 & Magic string `GBOR' \\
   4 &  4 & Version number of the block `1'. \\
   8 &  4 & Length of the block excluding magic string, version, and length. \\
  12 &  2 & Widget type. \\
  14 &  1 & Border width of the top edge. \\
  15 &  1 & Border width of the left edge. \\
  16 &  1 & Border width of the right edge. \\
  17 &  1 & Border width of the bottom edge. \\
  18 &  1 & Minimal width of the border. \\
  19 &  1 & Minimal height of the border. \\
  20 &  1 & Horizontal stepsize of the border. \\
  21 &  1 & Vertical stepsize of the border. \\
  22 &  4 & Top-left sprite. \\
  26 &  4 & Top-middle sprite. \\
  30 &  4 & Top-right sprite. \\
  34 &  4 & Left sprite. \\
  38 &  4 & Middle sprite. \\
  42 &  4 & Right sprite. \\
  46 &  4 & Bottom-left sprite. \\
  50 &  4 & Bottom-middle sprite. \\
  54 &  4 & Bottom-right sprite. \\
  58 & \multicolumn{2}{l|}{Total length} \\ \hline
\end{tabular}
\end{center}

Known widget types:
\begin{itemize}
\item 0 Invalid, do not use.
\item 16 Window border.
\item 32 Titlebar.
\item 48 button,
      49 pressed button,
      52 rounded button,
      53 pressed rounded button.
\item 64 frame
\item 80 inset frame
\end{itemize}

\subsection{Checkbox and radio buttons}
\begin{center}
\begin{tabular}{|r|c|l|} \hline
\textbf{Offset} & \textbf{Length} & \textbf{Contents description} \\ \hline
   0 &  4 & Magic string `GCHK' \\
   4 &  4 & Version number of the block `1'. \\
   8 &  4 & Length of the block excluding magic string, version, and length. \\
  12 &  2 & Widget type. \\
  14 &  4 & Empty. \\
  18 &  4 & Filled. \\
  22 &  4 & Empty pressed. \\
  26 &  4 & Filled pressed. \\
  30 &  4 & Shaded empty button. \\
  34 &  4 & Shaded filled button. \\
  38 & \multicolumn{2}{l|}{Total length} \\ \hline
\end{tabular}
\end{center}

Known widget types:
\begin{itemize}
\item 96 Checkbox.
\item 112 Radio-button.
\end{itemize}

\subsection{Slider-bar elements}
For slider-bar GUI elements, the following block
should be used.
\begin{center}
\begin{tabular}{|r|c|l|} \hline
\textbf{Offset} & \textbf{Length} & \textbf{Contents description} \\ \hline
   0 &  4 & Magic string `GSLI' \\
   4 &  4 & Version number of the block `1'. \\
   8 &  4 & Length of the block excluding magic string, version, and length. \\
  12 &  1 & Minimal length of the bar. \\
  13 &  1 & Stepsize of the bar. \\
  14 &  1 & Width of the slider button. \\
  15 &  2 & Widget type. \\
  17 &  4 & Left sprite. \\
  21 &  4 & Middle sprite. \\
  25 &  4 & Right sprite. \\
  29 &  4 & Slider button. \\
  33 & \multicolumn{2}{l|}{Total length} \\ \hline
\end{tabular}
\end{center}

Known slider-bar widget types:
\begin{itemize}
\item 128 Horizontal slider bar + button
\item 129 Shaded horizontal slider bar + button
\item 144 Vertical slider bar + button
\item 145 Shaded vertical slider bar + button
\end{itemize}


\subsection{Scroll-bar elements}
For scroll-bar GUI elements, the following block
should be used.
\begin{center}
\begin{tabular}{|r|c|l|} \hline
\textbf{Offset} & \textbf{Length} & \textbf{Contents description} \\ \hline
   0 &  4 & Magic string `GSCL' \\
   4 &  4 & Version number of the block `1'. \\
   8 &  4 & Length of the block excluding magic string, version, and length. \\
  12 &  1 & Minimal length scrollbar. \\
  13 &  1 & Stepsize of background. \\
  14 &  1 & Minimal length bar. \\
  15 &  1 & Stepsize of bar. \\
  16 &  2 & Widget type. \\
  18 &  4 & Left button. \\
  22 &  4 & Right button. \\
  26 &  4 & Left bar bottom (the background). \\
  30 &  4 & Middle bar bottom (the background). \\
  34 &  4 & Right bar bottom (the background). \\
  38 &  4 & Left bar top. \\
  42 &  4 & Middle bar top. \\
  46 &  4 & Right bar top. \\
  50 & \multicolumn{2}{l|}{Total length} \\ \hline
\end{tabular}
\end{center}

Known scroll-bar widget types:
\begin{itemize}
\item 160 Horizontal scroll bar + button
\item 161 Shaded horizontal scroll bar + button
\item 176 Vertical scroll bar + button (left = top, right = bottom).
\item 177 Shaded vertical scroll bar + button (left = top, right = bottom).
\end{itemize}


%%
%%
%%
\newpage
\section{Future}

To consider:
\begin{itemize}
\item Place for the license
\item Author and other information?
\item Readme document?
\item \ldots?
\end{itemize}

To do:
\begin{itemize}
\item (nothing, at the moment)
\end{itemize}

\end{document}
% 'a,'b ! count_offset

% vim: set spell
